\section{Cardinaux de quelques ensembles finis}
  \begin{description}
    \item[L'ensemble des $p$-uples de $\llbracket 1,n \rrbracket$]
      \[
        \#(\llbracket 1,n \rrbracket ^p) = n^p
      \]
      Globalement:
      \[
        \#(A^p) = (\#A)^p
      \]

      \item[L'ensemble des parties de $\llbracket 1,n \rrbracket$]
      Il est en bijection avec $\{0,1\}^n$. En effet, on a pour chacun des
      le choix ou non de le considérer dans une partie. Un $n$-uplet
      de $\{0,1\}^n$ correspond donc, à la partie où la $i$-ème coordonnée aura
      été prise si elle vaut $1$, sinon laissée. C'est donc une bijection.

      %Améliorer

      On a donc: \[
        \#\mathcal{P}(\llbracket 1,n \rrbracket) = 2^n
      \]

% FIN DU PREMIER AMPHI

      \item[Ensemble des permutations de $\llbracket 1,n \rrbracket$]
      Une permutation est une bijection $\sigma$ de $\llbracket 1,n \rrbracket$:
      \begin{eqnarray*}
        \sigma :  1,n  &\rightarrow&  1,n \\
                             i             &\mapsto& \sigma(i)\\
      \end{eqnarray*}
      $\forall i\neq j, f(i)\neq f(j)$

      Très souvent on note une permutation par son image:
      $
      \begin{pmatrix}
        1 & 2 & \dots & n\\
        \sigma(1) & \sigma(2) & \dots & \sigma(n)
      \end{pmatrix}
      $

      ou $(\sigma(1)\, \sigma(2)\, \dots\, \sigma(n))$.

      \[
        \mathfrak{S}_n = \{(x_1\, \dots\, x_n), x_i\in\llbracket 1,n\rrbracket, i\neq j \iff x_i\neq x_j\}\\
        \#\mathfrak{S}_n = n!
      \]
  \end{description}
