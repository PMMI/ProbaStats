\section{Cardinaux de quelques ensembles finis}
  \begin{description}
    \item[L'ensemble des $p$-uples de $\llbracket 1,n \rrbracket$]
      \[
        \#(\llbracket 1,n \rrbracket ^p) = n^p
      \]
      Globalement:
      \[
        \#(A^p) = (\#A)^p
      \]

      \item[L'ensemble des parties de $\llbracket 1,n \rrbracket$]
      Il est en bijection avec $\{0,1\}^n$. En effet, on a pour chacun des
      le choix ou non de le considérer dans une partie. Un $n$-uplet
      de $\{0,1\}^n$ correspond donc, à la partie où la $i$-ème coordonnée aura
      été prise si elle vaut $1$, sinon laissée. C'est donc une bijection.

      %Améliorer

      On a donc: \[
        \#\mathcal{P}(\llbracket 1,n \rrbracket) = 2^n
      \]
  \end{description}
