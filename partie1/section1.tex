\section{Cardinal d'un esnemble fini}
$\Omega$ un ensemble fini, $A\subset\Omega$.
\begin{definition}
  La fonction indicatrice de A est la fonction notée $1_A$, $\chi_A$ ou
  $\indic_A$ telle que:
  \begin{eqnarray*}
    \indic_A : \Omega &\rightarrow& \{0, 1\}\\
          \omega &\mapsto& \begin{cases}
                          1, \omega\in A\\
                          0, \omega\notin A
                        \end{cases}
  \end{eqnarray*}
\end{definition}

\begin{proposition}
  Deux sous ensembles $A$ et $B$ de $\Omega$ ont le même cardinal s'il existe
  une bijection entre $A$ et $B$.

  \[
    \# A = \# B \iff \forall x \in A,
    \exists ! y \in B, \exists f\in B^A | f(x)=y
  \]
  %Peut-être pas une bijection...

\end{proposition}

\subsection{Propriété des fonctions indicatrices}
\label{sub:Propriété des fonctions indicatrices}
  \begin{definition}
    Le complémentaire de $A$ dans $\Omega$ est l'ensemble des points des
    points de $\Omega$ qui ne sont pas dans $A$.

    \[A^C = \Omega\backslash A = \rdbrackets{\omega\in\Omega, \omega\notin A}
                               = \rdbrackets{\omega\in\Omega|\indic_A(\omega)=0}
    \]

  \end{definition}

  Une partie et son complémentaire forment une partition de $\Omega$.

  \begin{definition}
    On dit que $\parenthesis{A_i}_{i\in\Ninterval{1,k}}\in\mathcal{P}(\Omega)^k$
    est une partition de $\Omega$ si
    \begin{eqnarray*}
      \bigcup_{i=1}^k A_i = \Omega \text{ et }
      \forall i\neq j, A_i\cap A_j = \emptyset
    \end{eqnarray*}
  \end{definition}

  \begin{definition}
    Il existe $n\in\NN$ tel que $A$ est en bijection avec $\Ninterval{1,n}$.
    Cet entier $n$ est unique et on l'appelle le cardinal de $A$.
    Noté : $\#A$ .%ou $\card A$
  \end{definition}

  Cela signifie qu'il est possible de numéroter les éléments de $A$ par
  $\rdbrackets{a_1, \dots, a_n}$, et $A$ est par exemple donné par:
  \[ k \in \Ninterval{1, n} \mapsto a_k \in A\]

  \begin{proposition}
    Le cardinal de $A$ vaut:
    \[\#A = \sum_{\omega\in \Omega} \indic_A(\omega)\]

    Cette somme est indépendante de la manière de numéroter l'ensemble.
  \end{proposition}

  \begin{proposition}
    \begin{itemize}
      \item \[\indic_{A^C} = 1-\indic_A\]
      \item Si $A\cap B = \emptyset$, alors
            \[\indic_{A\cup B} = \indic_A + \indic_B\]
      \item Plus généralement:
            \[\indic_{A\cup B} = \indic_A + \indic_B - \indic_{A\cap B}\]
      \item \[\indic_{A\cap B} = \indic_A \times \indic_B\]
      \item \[\indic_{A\times B}(x,y) = \indic_A(x)\times \indic_B(y)\]
    \end{itemize}
  \end{proposition}

  \begin{corollaire}
    \begin{itemize}
      \item \[\#A^C = \#\Omega - \#A\]
      \item \[
        \#(A\cup B) = \#A + \#B - \#(A\cap B)
      \]
    \end{itemize}
  \end{corollaire}

  \begin{preuve}
    Soit $\omega\in\Omega$.\\
    Si $\omega\in A$ alors $\indic_{A^C}(\omega)=0$ et
       $1-\indic_{A^C}(\omega) = 1 = \indic_A(\omega)$.\\
    Si $\omega\notin A$, alors $\omega\in A^C$ et la démonstration est
    identique.\\
    Les deux fonctions $\indic_A$ et $1-\indic_A$ sont donc égales sur
    $\Omega$ et on a égalité.\\

    Trivialités. \'A recopier qd le temps sera présent.


  \end{preuve}
