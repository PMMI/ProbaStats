\section{Définitions}

  \subsection{L'univers}

  \begin{definition}[L'univers $\Omega$]
    C'est un ensemble qui rassemble toutes les informations liés à une expérience
    aléatoire donnée.

    On a: $\Omega\neq\emptyset$.

  \end{definition}

    \begin{exemple}
      On jette un dé à six faces.

      On peut prendre $\Omega = \rdbrackets{1,2,3,4,5,6} = \rdbrackets{\text{numéros de la face}}$.
      $\Omega$ n'est cependant pas unique.
    \end{exemple}

 %% Fin du cours 2 ??????????????????????????????????

    Grilles de loto, peut être gros, position d'une particule suivant un mouvement
    brownien ($\Omega = (\RR^3)^\RR, \#\Omega = \aleph_2$).

    Cadre du cours : $\#\Omega = k \in \NN$. \'Etude des probabilités discrètes.

  \begin{definition}[Les évènements]
    Ce sont les \emph{parties} ou \emph{sous-ensembles} de $\Omega$.
    \[
      \mathcal{P}(\Omega) = \rdbrackets{\text{l'ensemble des évènements de $\Omega$}}
    \]

    Deux évènements sont dits incompatibles si leurs ensembles associés sont
    disjoints.

  \end{definition}

  \begin{definition}[Partition de $\Omega$]
    C.f. \ref{partiton}.
  \end{definition}

  \subsection{Probabilité}
  Pour $\Omega$ fini, $\omega \in \Omega$, on peut noter la fréquence d'apparition
  de $\omega$.

  \begin{definition}[Germe de probabilité]
    On appelle \emph{germe de probabilité}, défini sur $\Omega$ (fini), une
    application: \[
        p : \Omega \rightarrow [0,1] | \sum_{\omega\in\Omega} p(\omega) = 1
    \]
  \end{definition}

  Autrement dit: \begin{eqnarray*}
    \forall \omega \in \Omega, 0 \leq p(\omega) \leq 1\\
    \sum_{\omega\in\Omega} p(\omega) = 1 = \sum_{i=1}^{\#\Omega} p(\omega_i)
  \end{eqnarray*}


  \begin{definition}
    La probabilité $\PP$ associée au germe de probabilité $p$ défini sur $\Omega$
    est l'application: \begin{eqnarray*}
      \PP : \mathcal{P}(\Omega) &\rightarrow& [0,1]\\
                     E         &\mapsto& \sum_{\omega\in E} p(\omega)
  \end{eqnarray*}
  \end{definition}

  \begin{remarque}
    \[
      \PP(\Omega) = 1
    \]
  \end{remarque}

  \begin{remarque}
    On a:
    \[
      \PP(E) = \sum_{\omega\in\Omega} \indic_E(\omega) p(\omega)
    \]
  \end{remarque}

  \begin{proposition}
    $A, B \in \mathcal{P}(\Omega)$, alors:
    \[\PP(A\cup B) = \PP(A) + \PP(B) - \PP(A\cap B)\]
    \[\PP(A^C) = 1 - \PP(A)\]
  \end{proposition}

  \begin{remarque}
    La proposition caractérise $\PP$ et $p$, i.e. si $\PP$ vérifie la proposition
    alors on sait définir le germe $p$ par:
    \[
      p(\omega) = \PP[\{\omega\}]
    \]
    et $\PP$ est unique.
  \end{remarque}

  \subsection{Espaces de probabilités}
    C'est la donnée d'un espace de probabilités $(\PP, \Omega)$ ou $(p,\Omega)$.

    \begin{definition}[Probabilité uniforme]
      Une \emph{probabilité uniforme} est la probabilité définie par le germe
      constant: \[
        \forall \omega \in \Omega, p(\omega) = \varpi,
        \sum_{\omega\in\Omega} p(\omega) = \#\Omega \times \varpi = 1 \iff \varpi = \frac{1}{\#\Omega}
      \]
    \end{definition}

    Dans le cas où la probabilité définie est uniforme, alors: \[
      \forall A\in\mathcal{P}(\Omega), \PP(A) = \frac{\#A}{\#\Omega}
    \]
