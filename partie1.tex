\part{Espaces de Probailité}
  Quelques notions de dénombrement
  \section{Cardinal d'un esnemble fini}
  $\Omega$ un ensemble fini, $A\subset\Omega$.
  \begin{definition}
    La fonction indicatrice de A est la fonction notée $1_A$, $\chi_A$ ou
    $\mathbb{1}_A$ telle que:
    \begin{eqnarray*}
      1_A : \Omega &\rightarrow& \{0, 1\}\\
            \omega &\mapsto& \begin{cases}
                            1, \omega\in A\\
                            0, \omega\notin A
                          \end{cases}
    \end{eqnarray*}
  \end{definition}

  \begin{proposition}
    Deux sous ensembles $A$ et $B$ de $\Omega$ ont le même cardinal s'il existe
    une bijection entre $A$ et $B$.

    \[\# A = \# B \iff \forall x \in A, \exists ! y \in B, \exists f\in B^A | f(x)=y\]
    %Peut-être pas une bijection...

  \end{proposition}

  \subsection{Propriété des fonctions indicatrices}
  \label{sub:Propriété des fonctions indicatrices}
    \begin{definition}
      Le complémentaire de $A$ dans $\Omega$ est l'ensemble des points des
      points de $\Omega$ qui ne sont pas dans $A$.

      \[A^C = ^CA = \Omega\backslash A = \{\omega\in\Omega, \omega\notin A\}
                                       = \{\omega\in\Omega|1_A(\omega)=0\}
      \]

    \end{definition}

    Une partie et son complémentaire forment une partition de $\Omega$.

    \begin{definition}
      On dit que $\left(A_i\right)_{i\in\llbracket 1,k\rrbracket} \in \mathcal{P}(\Omega)^k$
      est une partition de $\Omega$ si
      \begin{eqnarray*}
        \bigcup_{i=1}^k A_i = \Omega \text{ et } \forall i\neq j, A_i\cap A_j = \emptyset
      \end{eqnarray*}
    \end{definition}

    \begin{definition}
      Il existe $n\in\mathbb{N}$ tel que $A$ est en bijection avec $\llbracket 1,n \rrbracket$.
      Cet entier $n$ est unique et on l'appelle le cardinal de $A$.
      Noté : $\#A$ .%ou $\card A$
    \end{definition}

    Cela signifie qu'il est possible de numéroter les éléments de $A$ par $\{a_1, \dots, a_n\}$,
    et $A$ est par exemple donné par:
    \[ k \in \llbracket 1, n \rrbracket \mapsto a_k \in A\]

    \begin{proposition}
      Le cardinal de $A$ vaut:
      \[\#A = \sum_{\omega\in \Omega} 1_A(\omega)\]
    \end{proposition}
